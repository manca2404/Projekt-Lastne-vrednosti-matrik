\documentclass[12pt,a4paper]{article}
\usepackage[slovene]{babel}
\usepackage[utf8]{inputenc}
\usepackage[T1]{fontenc}
\usepackage{lmodern}
\usepackage{geometry}
\geometry{margin=2.5cm}
\usepackage{graphicx}
\usepackage{booktabs}
\usepackage{caption}
\usepackage{amsmath,amssymb}
\usepackage{float}
\usepackage[hidelinks]{hyperref}

\title{Circulantne matrike}
\author{Amanda Babič, Manca Kavčič \\ Fakulteta za matematiko in fiziko}
\date{\today}

\begin{document}

\maketitle

\begin{abstract}
Kratek povzetek poročila (cilji, metode, glavni rezultati). Naj bo 3–6 stavkov.
\end{abstract}

\section{Cirkulantne matrike}

\subsection{Definicija}
Cirkulantna matrika je posebna vrsta kvadratne matrike, pri kateri je vsaka vrstica dobljena s
\textbf{cikličnim premikom} elementov prejšnje vrstice za eno mesto v desno.
Podrobne definicije in izpeljave lastnosti so opisane v \cite{DavisCirculant1994, Gray2006Toeplitz, Pillai2020Circulant}.

Če je prva vrstica matrike
\[
(c_0, c_1, c_2, \ldots, c_{n-1}),
\]
potem ima splošna $n \times n$ cirkulantna matrika obliko:
\[
C =
\begin{pmatrix}
c_0 & c_1 & c_2 & \cdots & c_{n-1} \\
c_{n-1} & c_0 & c_1 & \cdots & c_{n-2} \\
c_{n-2} & c_{n-1} & c_0 & \cdots & c_{n-3} \\
\vdots & \ddots & \ddots & \ddots & \vdots \\
c_1 & c_2 & c_3 & \cdots & c_0
\end{pmatrix}.
\]
Vsaka cirkulantna matrika je torej popolnoma določena s svojo prvo vrstico.

\subsection{Lastnosti cirkulantnih matrik}

\begin{itemize}
    \item Cirkulantne matrike tvorijo podprostor prostora vseh $n \times n$ matrik \cite{Gray2006Toeplitz}.
    \item Produkt, vsota in potenca cirkulantnih matrik so spet cirkulantne matrike.
    \item Vse cirkulantne matrike imajo iste lastne vektorje; razlikujejo se le po lastnih vrednostih \cite{DavisCirculant1994}.
\end{itemize}

\subsection{Lastni vektorji in lastne vrednosti}

Ena najpomembnejših lastnosti cirkulantnih matrik je, da so njihove lastne vrednosti in vektorji
tesno povezani z diskretno Fourierjevo transformacijo (DFT), kar poudarjata \cite{Pillai2020Circulant, Kumar2019ComputationPerspective}.

Naj bo $\omega_n = e^{\frac{2\pi i}{n}}$ primitivna $n$-ta korenina enote.

Za $k = 0, 1, \ldots, n-1$ definiramo lastne vektorje:
\[
x^{(k)} =
\begin{pmatrix}
1 \\ \omega_n^k \\ \omega_n^{2k} \\ \vdots \\ \omega_n^{(n-1)k}
\end{pmatrix}.
\]
Lastne vrednosti $\lambda_k$ dobimo kot diskretno Fourierjevo transformacijo prve vrstice matrike:
\[
\lambda_k = \sum_{j=0}^{n-1} c_j \, \omega_n^{jk}, \qquad k = 0, 1, \ldots, n-1.
\]
S tem lahko vsako cirkulantno matriko zapišemo v diagonalni obliki:
\[
C = F^{-1} \Lambda F,
\]
kjer je $F$ Fourierjeva matrika z elementi
\[
F_{jk} = \omega_n^{jk}, \quad 0 \leq j,k \leq n-1,
\]
in $\Lambda = \text{diag}(\lambda_0, \lambda_1, \ldots, \lambda_{n-1})$.

Fourierjeva matrika $F$ je unitarna:
\[
F^{-1} = \frac{1}{n} F^H,
\]
kar pomeni, da imajo cirkulantne matrike \textbf{ortogonalne lastne vektorje} \cite{Gray2006Toeplitz, Hariprasad2024CirculantDecomposition}.

\subsection{Povezava s Fourierjevo transformacijo}

Iz zgornjih rezultatov sledi, da je množenje vektorja $x$ s cirkulantno matriko $C$ ekvivalentno
\textbf{konvoluciji} med prvim stolpcem matrike in vektorjem $x$:
\[
Cx = c * x,
\]
kar pomeni, da se lahko takšne operacije izvedejo učinkovito z uporabo
\textbf{hitre Fourierjeve transformacije (FFT)} v času $O(n \log n)$ \cite{Hariprasad2024CirculantDecomposition}.

\subsection{Opomba}
Če so elementi prve vrstice $c_j$ realni, potem se lastne vrednosti pojavijo v konjugiranih parih,
torej $\lambda_k = \overline{\lambda_{n-k}}$.
Realni in imaginarni deli lastnih vektorjev ustrezajo diskretni kosinusni in sinusni transformaciji (DCT, DST)
\cite{Pillai2020Circulant, Kumar2019ComputationPerspective}.

\subsection{Primer}
Oglejmo si cirkulantno matriko reda $4$ z prvo vrstico
\[
(c_0, c_1, c_2, c_3) = (2, -1, 0, -1).
\]
Dobimo:
\[
C =
\begin{pmatrix}
2 & -1 & 0 & -1 \\
-1 & 2 & -1 & 0 \\
0 & -1 & 2 & -1 \\
-1 & 0 & -1 & 2
\end{pmatrix}.
\]
Za $n=4$ velja $\omega_4 = e^{\frac{2\pi i}{4}} = i$.
Lastne vrednosti so:
\[
\begin{aligned}
\lambda_0 &= 2 + (-1) + 0 + (-1) = 0, \\
\lambda_1 &= 2 - i(1) + 0 + i(1) = 2, \\
\lambda_2 &= 2 - (-1) + 0 - (-1) = 4, \\
\lambda_3 &= 2 + i(1) + 0 - i(1) = 2.
\end{aligned}
\]
Torej so lastne vrednosti matrike $C$:
\[
\lambda = (0, 2, 4, 2).
\]
Pripadajoči lastni vektorji so stolpci Fourierjeve matrike:
\[
F =
\begin{pmatrix}
1 & 1 & 1 & 1 \\
1 & i & -1 & -i \\
1 & -1 & 1 & -1 \\
1 & -i & -1 & i
\end{pmatrix}.
\]
Zato velja:
\[
C = F^{-1}
\begin{pmatrix}
0 & 0 & 0 & 0 \\
0 & 2 & 0 & 0 \\
0 & 0 & 4 & 0 \\
0 & 0 & 0 & 2
\end{pmatrix} F.
\]
S tem je matrika diagonalizirana, kar potrjuje, da imajo vse cirkulantne matrike enake lastne vektorje (Fourierjevo bazo),
njihove lastne vrednosti pa ustrezajo diskretni Fourierjevi transformaciji prve vrstice \cite{Gray2006Toeplitz, Hariprasad2024CirculantDecomposition}.
\vspace{1em}

\bibliographystyle{plain}
\bibliography{references}

\end{document}